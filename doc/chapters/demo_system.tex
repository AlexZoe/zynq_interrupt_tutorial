\section{Hardware Timer Example}

\subsection{Introduction}\label{demo-intro}
This section describes the demonstrator system which uses a timer in programmable logic for generating interrupts on the Xilinx Zynq platform.


\subsection{Prerequisites}
The demonstrator system has been implemented on the Digilent Zybo Z7-20 board with Xilinx Vivado 2018-3.
If you want to use the demonstrator system directly, the aforementioned board is required with a Vivado installation containing the board support package.
Any Vivado installation \emph{should} work.
If you use a Zybo Z7 board, a Vivado version 2018.x and later is more convenient since you will not have to backport devicetrees and menuconfig when building the Linux image.
See Appendix~\ref{Vivado-bitstream} if you want to use a different board or are interested in how to manually build the system on the FPGA for generating the bitstream.

Debian 9 with Linux kernel 4.19 is used as the Linux image on the Zybo board, which boots from a micro SD card.
The image has been built using this \cite{DebianImage} guide.
See Appendix~\ref{Linux-Image} for details regarding the Linux image.

The prerequisites can be summarized as follows:
\begin{itemize}
\item Vivado installation (version 2017.x and later /emph{should} work)
\item A evaluation board with a Zynq 7000 chip (e.g. ZedBoard, MicroZed, Zybo, Zybo Z7)
\item Board support package for the board to be used in Vivado
\end{itemize}

\subsection{Hardware Implementation}


\subsection{Linux Kernel}\label{kernel}
%"Documentation/devicetree/bindings/interrupt-controller/arm,gic.yaml"

%kernel version: 4.19
% commit on linux-xlnx: 0ee2ead168af335315b6cd8c55cbf6e45998f585
% commit on u-boot-xlnx: 195c620e348891ca2d90c759781413f2adb3f748

\subsection{Linux User Space}
