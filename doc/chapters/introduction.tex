\section{Introduction}
Interrupts are an essential part for communication between hardware and software.
Using interrupts prevents wasting precious CPU clock cycles on polling as well as improving the systems' responsiveness.

The Xilinx Zynq 7000 system-on-chip (SoC) series offers a number of interrupt ports between programmable logic (PL) - the FPGA side - and the processing system (PS) - the CPU.
This requires a logic implementation on the FPGA side, which can be either one of the interrupt capable IP cores of Xilinx or a custom logic.
On the CPU(s), software for configuring and receiving interrupts has to be executed.
This software can be run bare-metal (without an operating system) or with an operating system such as Linux.
The former is well covered by online tutorials and is pretty straight forward to get a working system.
Accomplishing the same using Linux, however, is not that well covered despite being an integral part even when using trivial logic such as timers.
\newline

Since it required quite some effort collecting information from user manuals, forum posts and online documentation in order to get interrupts to work with Linux, I would like to spare others (and myself in the future) the trouble of having to repeat the same all over.
\newline

%This guide is split into two sections.
Section~\ref{quick} is for the impatient and those who just want to get the job done without caring too much about the details.
Therefore, presented information are kept to a minimum and as generally applicable to other systems as possible.
A basic understanding of the work flow with Xilinx's Vivado, compiling a Linux image for a particular board and basic understanding of Linux device drivers are required.

%The second part of this guide, Section~\ref{example}, is for those who want a more detailed explanation (or are new to the Zynq platform or Linux device drivers).
%A basic example system is provided which demonstrates generating interrupt events using a timer in programmable logic and a device driver for processing the interrupt events.
